\documentclass[11pt]{article}

\usepackage{amsmath,amssymb}
\usepackage{graphicx}
\usepackage{hyperref}
\usepackage{geometry}
\geometry{margin=1in}

\title{The God Variable: A Constraint-Based Exploration of Long-Horizon Survivability}
\author{William Shacklett}
\date{\today}

\begin{document}

\maketitle

\begin{abstract}
Complex systems often fail not through abrupt malfunction, but through gradual erosion of recovery paths and the accumulation of irreversible changes. Existing analytical and engineering frameworks tend to emphasize short-horizon correctness, leaving long-term survivability and self-correctability underexamined.

This manuscript explores a constraint-based modeling approach centered on a proposed scalar quantity, the \emph{God Variable} (GV), intended to summarize a system’s capacity for long-horizon survival, self-correction, and resistance to irreversible degradation. GV is not assumed to be fundamental, conserved, or universal; rather, it is treated as a candidate constraint measure whose utility must be demonstrated empirically.

We present a set of toy models and simulations that operationalize GV in simplified settings, examining its internal consistency, sensitivity to perturbations, and behavior under cumulative system evolution. These models are used to test whether scalar survivability constraints can meaningfully detect degradation pathways that evade conventional metrics.

The contribution of this work is exploratory. By framing survivability as a constraint rather than an objective, the manuscript aims to clarify what such a scalar can and cannot capture, to identify limitations inherent in single-metric formulations, and to outline pathways for falsification and extension. Negative results are treated as informative, with the goal of establishing a disciplined foundation for further investigation rather than asserting a complete or universal theory.
\end{abstract}

\section{Introduction}

Many complex systems exhibit failure modes that are not attributable to single catastrophic events, but instead arise through the gradual loss of recovery mechanisms, increasing brittleness, and the accumulation of irreversible changes. Software systems, biological processes, and physical models alike often appear stable under short-horizon evaluation while drifting toward states from which recovery becomes increasingly unlikely.

Tr
